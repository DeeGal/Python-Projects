# Diana Mudau

input_numbers = [1, 3, 5, 3, 7, 2, 2, 3, 2, 1, 3, 5]  # 1D number array.
count = 0  # The list count starts at the first number in the array.
index_val = 0  # The list index starts at the first number in the array.
sel_value = 0  # The values selected begin from the first value in the array to the last value.

for numbers in range(0, len(input_numbers)):  # For the available numbers selected in the array
    sel_value = input_numbers.count(input_numbers[numbers])  # Count all the numbers in the array
    
    if sel_value > count:  # select the most frequent number in the array
        count = sel_value  # How many times the selected value was counted.
        index_val = numbers  # How many times the index of the number in the array was counted.

most_freq_number = input_numbers[index_val]  # Function Declaration.
print("The most frequent number is : ", most_freq_number)  # Display the number.
print("The number appeared ", count, "time(s) in the list.")  # Display how many times the number appeared the array.


for numbers in range(0, len(input_numbers)):  # For the available numbers selected in the array
    sel_value = input_numbers.count(input_numbers[numbers])  # Count all the numbers in the array

    if sel_value < count:  # select the least frequent number in the array
        count = sel_value  # How many times the selected value was counted.
        index_val = numbers  # How many times the index of the number in the array was counted.

least_freq_number = input_numbers[index_val]  # Function Declaration.
print("The least frequent number is : ", least_freq_number)  # Display the number.
print("The number appeared ", count, "time(s) in the list.")  # Display how many times the number appeared the array.

